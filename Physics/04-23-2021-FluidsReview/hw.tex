\documentclass{article}
\usepackage{amsmath}
\usepackage[smartEllipses,hybrid]{markdown}
\usepackage[margin=0.5in]{geometry}
\usepackage{enumitem}
\usepackage{hyperref}
% \setlist[itemize]{align=parleft,left=0pt..1em}
% \setlist[enumerate]{align=parleft,left=0pt..1em}
\renewcommand{\labelitemii}{$\circ$}
\renewcommand{\labelitemiii}{$\circ$}

\title{Physics Homework}
\author{Neo Wang}
\date{\today}

\begin{document}

\maketitle
\tableofcontents

\section{A}

\subsection{i} Student $Y$'s claim that the speed at point $B$ is greater than at point $A$ is correct.

\subsection{ii} The claim that the pressure is greater because of the water speed is incorrect. In fact, pressure can be affected by external forces, such as gravity. Since the elevation at point $B$ is higher, then pressure must be less.

\subsection{iii}

Student $Z$ is right in the regard that conversation of energy converts some of the kinetic energy into potential energy. However, this does not necessarily mean that the stream is slower.

\subsection{iv}

Student $Z$ doesn't realize that just because the water goes up doesn't mean that it can't be faster. In that sense, it wouldn't make sense for a hose to work, as the water clearly exits at a speed faster than when it enters.

\section{B}

\subsection{i}

$$(2.5)^2(0.5)/(1.5^2)=1.4 m/s$$

\subsection{ii}

$$P_B = 2\cdot 10^5 + (1000)(10)(-5) + \frac{1}{2}(1000)(0.5^2-1.4^2) = 1.5\cdot 10^5 Pa$$

\section{C}

\subsection{i}

$$P_A = P_B + \rho gh = 10^5 + 60000 = 1.6 \cdot 10^5 Pa$$

\subsection{ii}

There should be a vector of length two facing upwards labeled $F_B$. There should be a vector of length $1$ labeled $F_g$.

\end{document}