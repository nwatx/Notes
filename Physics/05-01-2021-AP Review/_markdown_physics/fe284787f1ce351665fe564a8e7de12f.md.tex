\markdownRendererHeadingOne{Definitions}\markdownRendererInterblockSeparator
{}\markdownRendererUlBeginTight
\markdownRendererUlItem Newton's third law - Don't forget that this requires there to be two different objects.\markdownRendererUlItemEnd 
\markdownRendererUlItem Electric fields will point in the same direction if $q$ is positive.\markdownRendererUlItemEnd 
\markdownRendererUlEndTight \markdownRendererInterblockSeparator
{}\markdownRendererHeadingOne{General Tips}\markdownRendererInterblockSeparator
{}\markdownRendererUlBeginTight
\markdownRendererUlItem When writing experimental lab procedures, \markdownRendererStrongEmphasis{ALWAYS MAKE SURE TO INCLUDE MULTIPLE EXPERIMENTAL TRIALS}. This almost always gurantees an extra point.\markdownRendererUlItemEnd 
\markdownRendererUlItem Be careful when deriving equations, it can almost always be observed that there are less than 3 major steps/concepts in the derivation. Any more, and you are likely overcomplicating things.\markdownRendererUlItemEnd 
\markdownRendererUlItem Important concept: The mechanical energy of a system is conserved as long as no force is acting on or acted upon by the system.\markdownRendererUlItemEnd 
\markdownRendererUlItem Important concept: Newton third-law pairs. Also, this requires for there to be a \markdownRendererStrongEmphasis{pair} (meaning two or more elements).\markdownRendererUlItemEnd 
\markdownRendererUlItem Make sure to reference your equation sheet when you don't know something; oftentimes, this results in the answer.\markdownRendererUlItemEnd 
\markdownRendererUlItem Remember to fake-solve problems.\markdownRendererUlItemEnd 
\markdownRendererUlEndTight \markdownRendererInterblockSeparator
{}\markdownRendererHeadingOne{Buoyancy}\markdownRendererInterblockSeparator
{}\markdownRendererUlBeginTight
\markdownRendererUlItem The force of buoyancy is the mass of the fluid displaced in the liquid.\markdownRendererUlItemEnd 
\markdownRendererUlEndTight \markdownRendererInterblockSeparator
{}\markdownRendererHeadingOne{Electric Fields}\markdownRendererInterblockSeparator
{}\markdownRendererUlBegin
\markdownRendererUlItem We can think of electric fields as an area permeated space where any mass placed in this field would experience a force because of an interaction with the field.\markdownRendererUlItemEnd 
\markdownRendererUlItem Electric field lines always point away from positive source charges and towards negative ones. Such pairs are called \markdownRendererStrongEmphasis{electric dipoles}.\markdownRendererUlItemEnd 
\markdownRendererUlEnd \markdownRendererInterblockSeparator
{}$$ F_E = qE $$\markdownRendererInterblockSeparator
{}\markdownRendererUlBegin
\markdownRendererUlItem The work energy theorem $$\Delta W = \frac{1}{2}mv^2$$\markdownRendererUlItemEnd 
\markdownRendererUlItem Please don't forget that the electric field is given by $$E = \frac{F}{Q}$$\markdownRendererUlItemEnd 
\markdownRendererUlEnd \markdownRendererInterblockSeparator
{}\markdownRendererHeadingTwo{Uniform Electric Fields}\markdownRendererInterblockSeparator
{}For all practical purposes, the field near the middle is uniform. Therefore, you can use kinematics just like we did near the surface of the earth.\markdownRendererInterblockSeparator
{}\markdownRendererHeadingTwo{Condutors and Insulators}\markdownRendererInterblockSeparator
{}\markdownRendererUlBeginTight
\markdownRendererUlItem The electric field inside a conductor is $0$.\markdownRendererUlItemEnd 
\markdownRendererUlItem Any excess charge on a conductor resides entirely on the outer surface.\markdownRendererUlItemEnd 
\markdownRendererUlItem When doing calculations with other charges outside of the metal ball, treat the metal ball as a singular charge, with its charge being centered, and having a value of $\sum outer$.\markdownRendererUlItemEnd 
\markdownRendererUlItem You can induce a charge inside a conductor by bringing a charge closer.\markdownRendererUlItemEnd 
\markdownRendererUlEndTight \markdownRendererInterblockSeparator
{}\markdownRendererHeadingOne{Thermodynamics}\markdownRendererInterblockSeparator
{}\markdownRendererUlBeginTight
\markdownRendererUlItem Recall that the work done is the area under a curve for a PV diagram. Therefore, you can make optimizations by either first increasing $P$ or $V$, depending on what the question is asking.\markdownRendererUlItemEnd 
\markdownRendererUlItem Unlike other forces, the work done by $PV$ diagrams is \markdownRendererStrongEmphasis{not conservative} this means that it is \markdownRendererStrongEmphasis{path-dependent}, which further emphasizes the points written above.\markdownRendererUlItemEnd 
\markdownRendererUlEndTight \markdownRendererInterblockSeparator
{}\markdownRendererHeadingOne{Mistakes}\markdownRendererInterblockSeparator
{}\markdownRendererHeadingTwo{Practice 1}\markdownRendererInterblockSeparator
{}\markdownRendererUlBeginTight
\markdownRendererUlItem In Coloumbs law, recall that\markdownRendererUlItemEnd 
\markdownRendererUlEndTight \relax