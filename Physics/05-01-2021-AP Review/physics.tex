\documentclass{article}
\usepackage{amsmath}
\usepackage[smartEllipses,hybrid]{markdown}
\usepackage[margin=0.5in]{geometry}
\usepackage{enumitem}
\usepackage{hyperref}
\usepackage{parskip}
% \setlist[itemize]{align=parleft,left=0pt..1em}
% \setlist[enumerate]{align=parleft,left=0pt..1em}
\renewcommand{\labelitemii}{$\circ$}
\renewcommand{\labelitemiii}{$\circ$}

\title{Physics AP Review}
\author{Neo Wang}
\date{\today}

\begin{document}

\maketitle
\tableofcontents

\begin{markdown}

# Definitions

Newton's third law - Don't forget that this requires there to be two different objects.

# Electric Fields

* We can think of electric fields as an area permeated space where any mass placed in this field would experience a force because of an interaction with the field.

* Electric field lines always point away from positive source charges and towards negative ones. Such pairs are called **electric dipoles**.

$$
F_E = qE
$$

* The work energy theorem $$\Delta W = \frac{1}{2}mv^2$$

* Please don't forget that the electric field is given by $$E = \frac{F}{Q}$$

## Uniform Electric Fields

For all practical purposes, the field near the middle is uniform. Therefore, you can use kinematics just like we did near the surface of the earth.

## Condutors and Insulators

* The electric field inside a conductor is $0$.
* Any excess charge on a conductor resides entirely on the outer surface.
* When doing calculations with other charges outside of the metal ball, treat the metal ball as a singular charge, with its charge being centered, and having a value of $\sum outer$.
* You can induce a charge inside a conductor by bringing a charge closer. 

# Thermodynamics

* Recall that the work done is the area under a curve for a PV diagram. Therefore, you can make optimizations by either first increasing $P$ or $V$, depending on what the question is asking.
* Unlike other forces, the work done by $PV$ diagrams is **not conservative** this means that it is **path-dependent**, which further emphasizes the points written above.

# Mistakes

## Practice 1

* In Coloumbs law, recall that

\end{markdown}

$$
F = k\frac{q_1 q_2}{r^2}
$$

\begin{markdown}

instead of incorrectly forgetting the charges and $k$.

* Equipotential lines are perpendicular to the force lines.
* When asked for which of the following photons could not be absorbed, failed to recall that as long as the eV difference can correspond to any of the differences, or one of the levels themselves, then the photon can be observed.
* Consult the formulas sheet! $U_C = \frac{1}{2}CV^2$
* The inertial mass is given by the equation $$a = \frac{F}{m}$$

\end{markdown}

\end{document}