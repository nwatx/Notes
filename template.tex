\documentclass{article}
\usepackage{amsmath}
\usepackage[smartEllipses,hybrid]{markdown}
\usepackage[margin=0.5in]{geometry}
\usepackage{enumitem}
\usepackage{hyperref}
% \setlist[itemize]{align=parleft,left=0pt..1em}
% \setlist[enumerate]{align=parleft,left=0pt..1em}
\renewcommand{\labelitemii}{$\circ$}
\renewcommand{\labelitemiii}{$\circ$}

\title{Notes Template}
\author{Neo Wang}
\date{\today}

\begin{document}

\maketitle
\tableofcontents

\begin{markdown}

# General Document Features

* This is fairly simple, although the main feature is the **markdown** engine.

* We can use $inline$ expressions and equations like $x^2-5ax+c$ and $$\int e^x$$

* Everything from standard *italics* to **bold** to [links](https://nwatx.me) is supported. `There is also inbuild support for inline code.`

# General File Structure Tips

Since we want to be organized, there's several ways to organize your document. The preferred method is as follows:

## General Organization From: Project Folder ex. Linear Algebra

\begin{itemize}
    \item main.tex - cumulative, master document (think main class in programming language).
    \item Lessons folder. Should be in format ISO-Title, such as 2021-04-20-Gaussian Elimination.tex
    \item Reference folder. Should be for things like equation sheets, references.
    \item Projects folder. Anything project related.
\end{itemize}

\end{markdown}

\end{document}