A reading on Medium can be found \markdownRendererLink{here}{https://blog.statsbot.co/singular-value-decomposition-tutorial-52c695315254}{https://blog.statsbot.co/singular-value-decomposition-tutorial-52c695315254}{}. Now we will get into the SVD itself. Since the columns of $V$ are orthonormal\markdownRendererEllipsis{}\markdownRendererInterblockSeparator
{}$$VV^T=I$$\markdownRendererInterblockSeparator
{}$$A = U\sum V^T$$\markdownRendererInterblockSeparator
{}Where do we get the columns of $V$ and $U$ from?\markdownRendererInterblockSeparator
{}$$A^T = (U\sum U^T)^T=V\sum^2 V^T$$\markdownRendererInterblockSeparator
{}diagnolization of $A^TA$, so the $\sigma=\sqrt{\text{eigenvalues of}A^TA}$\relax