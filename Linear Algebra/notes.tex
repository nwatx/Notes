\documentclass{article}
\usepackage{amsmath}
\usepackage[smartEllipses,hybrid]{markdown}
\usepackage[margin=0.5in]{geometry}
\usepackage{enumitem}
\usepackage{hyperref}
\usepackage{parskip}
% \setlist[itemize]{align=parleft,left=0pt..1em}
% \setlist[enumerate]{align=parleft,left=0pt..1em}
\renewcommand{\labelitemii}{$\circ$}
\renewcommand{\labelitemiii}{$\circ$}

\title{Notes Template}
\author{Neo Wang}
\date{\today}

\begin{document}

\maketitle
\tableofcontents


\section{Single Value Decomposition}

\begin{markdown}

A reading on Medium can be found [here](https://blog.statsbot.co/singular-value-decomposition-tutorial-52c695315254). Now we will get into the SVD itself. Since the columns of $V$ are orthonormal...

$$VV^T=I$$

$$A = U\sum V^T$$

Where do we get the columns of $V$ and $U$ from?

$$A^T = (U\sum U^T)^T=V\sum^2 V^T$$

diagnolization of $A^TA$, so the $\sigma=\sqrt{\text{eigenvalues of}A^TA}$

\end{markdown}

After we find the $v$'s there is a shortcut for the $u$'s.

$$Av_i=\sigma_i\vec{u}_i$$

Example 1: Find the SVD for $A = \begin{bmatrix}
	3 & 0 \\4&5
\end{bmatrix}$

\begin{itemize}
	\item We can find $\sigma = \sqrt{5}, 3\sqrt{5}$ and eigenvalues of $5, 45$
	\item The full worked example can be found in your notes.
\end{itemize}

Example 2: Using the left SVD to find SVD of:

$$
\begin{aligned}
	A    & =
	\begin{bmatrix}
		0 & 1 & 0 & 0 \\0&0&2&0\\0&0&0&3
	\end{bmatrix} \\
	A^TA & =
	\begin{bmatrix}
		0 & 0 & 0 & 0 \\0&1&0&0\\0&0&4&0\\0&0&0&9
	\end{bmatrix} \\
\end{aligned}
$$

\begin{itemize}
	\item Since this matrix is diagonal, the eigenvalues are $0, 1, 4, 9$. Then $\sigma=0,1,2,3$ respectively.
\end{itemize}

Example 3 (from MIT OCW):

Find the singular value decomposition of $$A=\begin{bmatrix}
		4  & 4 \\
		-3 & 3
\end{bmatrix}$$

Step 1: We calculate

$$
A^TA=\begin{bmatrix}
	25 & 7 \\ 
	7 & 25
\end{bmatrix}
$$

Step 2: Find the eigenvectors of this matrix to get the vectors $v_i$, and the eigenvalues for $\sigma_i$.

For our example, the two orthogonal eigenvectors are $(1, 1)$ and $(1, -1)$. Our orthonormal basis, then is

$$
v_1=\left(\frac{1}{\sqrt{2}}, \frac{1}{\sqrt{2}}\right), v_2=\left(\frac{1}{\sqrt{2}}, -\frac{1}{\sqrt{2}}\right)
$$

\end{document}