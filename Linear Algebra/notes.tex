\documentclass{article}
\usepackage{amsmath}
\usepackage[smartEllipses,hybrid]{markdown}
\usepackage[margin=0.5in]{geometry}
\usepackage{enumitem}
\usepackage{hyperref}
% \setlist[itemize]{align=parleft,left=0pt..1em}
% \setlist[enumerate]{align=parleft,left=0pt..1em}
\renewcommand{\labelitemii}{$\circ$}
\renewcommand{\labelitemiii}{$\circ$}

\title{Notes Template}
\author{Neo Wang}
\date{\today}

\begin{document}

\maketitle
\tableofcontents

\begin{markdown}

# Single Value Decomposition

Since the columns of $V$ are orthonormal...

$$VV^T=I$$

$$A = U\sum V^T$$

Where do we get the columns of $V$ and $U$ from?

$$A^T = (U\sum U^T)^T$$

$=V\sum^2 V^T$, diagnolization of $A^TA$, so the $\sigma=\sqrt{\text{eigenvalues of}A^TA}$

\end{markdown}

After we find the $v$'s there is a shortcut for the $u$'s.

$$Av_i=\sigma_i\vec{u}_i$$

Example 1: Find the SVD for $A = \begin{bmatrix}
    3&0\\4&5
\end{bmatrix}$

\begin{itemize}
\item We can find $\sigma = \sqrt{5}, 3\sqrt{5}$ and eigenvalues of $5, 45$
\item The full worked example can be found in your notes.
\end{itemize}

Example 2: Using the left SVD to find SVD of:

$$
\begin{align*}
    A &= 
    \begin{bmatrix}
        0&1&0&0\\0&0&2&0\\0&0&0&3
    \end{bmatrix} \\
    A^TA &=
    \begin{bmatrix}
    0&0&0&0\\0&1&0&0\\0&0&4&0\\0&0&0&9
    \end{bmatrix} \\
\end{align*}
$$

\begin{itemize}
    \item Since this matrix is diagonal, the eigenvalues are $0, 1, 4, 9$. Then $\sigma=0,1,2,3$ respectively.
\end{itemize}

\end{document}