\documentclass{article}
\usepackage{amsmath}

\title{Argument Body Paragraph Notes}
\author{Neo Wang}
\date{\today}

\begin{document}

\maketitle

\begin{itemize}
    \item Be specific, instead of "pacific," where pacific ocean means broad.
    \item Talks about how individuals can find almost anything offensive, so it's unreliable to label everything as offensive.
    \item Sections into several movement, and concedes various points that in some cases its easy and plausible to label items as offensive.
    \item Template: (Transitional Language) (Somewhat specific reference to your example) + (an argument (abstraction) that in some way relates to your argument/thesis).
    \item There's no science to this; the order can change.
    \item Many ways to transition: concede, analysis.
    \item Topic sentence should have some sort of transitional language; helps boost score because it implies understanding.
    \item Commentary mostly about the same thing.
    \item Uses commentary like "just about anyone can admit that."
    \item In general, finds things that people can commonly agree on; there are little to no controversial statements in his argument.
    \item His paper is going to feel objective; holistic.
    \item Introduces a template in the format of reasoning, evidence.
    \item Take 3-5 sentences to describe your specific evidence, but the commentary should be at least 3-5 sentences.
    \item Points back to previous that is previously described. Does it disprove it? Does it prove it? Does it show the limitations of it?
    \item Use the word "because" or any variant of it. And the word "thus."
    \item This is very good for articulating your commentary.
\end{itemize}

\end{document}