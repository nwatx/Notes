\markdownRendererHeadingOne{Mistakes in Practice}\markdownRendererInterblockSeparator
{}\markdownRendererUlBeginTight
\markdownRendererUlItem There is close to no effect on twins being raised in different families; they can be just as similar as if they were raised together.\markdownRendererUlItemEnd 
\markdownRendererUlItem Cortisol is released by people when they are stressed.\markdownRendererUlItemEnd 
\markdownRendererUlItem The neuron goes from being negatively charged to briefly being positively charged, and finally returns to being negatively charged again. The magnitude of the negative charge is fixed regardless of the strength of the input signal it receives.\markdownRendererUlItemEnd 
\markdownRendererUlItem Retrieving objects takes place in the left hemisphere for most people.\markdownRendererUlItemEnd 
\markdownRendererUlItem Temporal lobe plays a role in object recognition and the parietal lobe plays a role in spatial processing.\markdownRendererUlItemEnd 
\markdownRendererUlItem REM sleep occurs only after all the sleep stages have been cycled through.\markdownRendererUlItemEnd 
\markdownRendererUlItem Four senses: Pressure, cold, warm, pain.\markdownRendererUlItemEnd 
\markdownRendererUlItem Explicit memories are created in the hippocampus.\markdownRendererUlItemEnd 
\markdownRendererUlItem Executive functioning is the the measure of working memory capacity.\markdownRendererUlItemEnd 
\markdownRendererUlItem Cortisol and stress are most closely related. The experiment done was\markdownRendererUlItemEnd 
\markdownRendererUlEndTight \markdownRendererInterblockSeparator
{}\markdownRendererHeadingOne{Research and History}\markdownRendererInterblockSeparator
{}\markdownRendererHeadingTwo{Psychoanalytic Approach}\markdownRendererInterblockSeparator
{}\markdownRendererUlBeginTight
\markdownRendererUlItem Very influenced by Sigmeund Freud.\markdownRendererUlItemEnd 
\markdownRendererUlItem In the late 1800s, he came up with a psychoanalytic approach which was about the unconscious drives.\markdownRendererUlItemEnd 
\markdownRendererUlItem For the most part, psychological approach were about drawing apart these unconscious desires.\markdownRendererUlItemEnd 
\markdownRendererUlEndTight \markdownRendererInterblockSeparator
{}\markdownRendererHeadingTwo{Behvaioral}\markdownRendererInterblockSeparator
{}\markdownRendererUlBeginTight
\markdownRendererUlItem Later, Ivan Pavlov, Skinner, Watson, theorized that you were all influenced about the ways that you were reinforced/punished.\markdownRendererUlItemEnd 
\markdownRendererUlEndTight \markdownRendererInterblockSeparator
{}\markdownRendererHeadingTwo{Humanist}\markdownRendererInterblockSeparator
{}\markdownRendererUlBeginTight
\markdownRendererUlItem Stemmed from hippies.\markdownRendererUlItemEnd 
\markdownRendererUlItem Humanist is the drive for free will.\markdownRendererUlItemEnd 
\markdownRendererUlItem You are trying to drive to reach your greatest free potential.\markdownRendererUlItemEnd 
\markdownRendererUlEndTight \markdownRendererInterblockSeparator
{}\markdownRendererBlockQuoteBegin
Mr. Rogers, isn't it a special day today
\markdownRendererBlockQuoteEnd \markdownRendererInterblockSeparator
{}\markdownRendererUlBeginTight
\markdownRendererUlItem Things like uncoditional positive regard, you are the best person to solve your own problem.\markdownRendererUlItemEnd 
\markdownRendererUlEndTight \markdownRendererInterblockSeparator
{}\markdownRendererHeadingTwo{Biological}\markdownRendererInterblockSeparator
{}\markdownRendererUlBeginTight
\markdownRendererUlItem Brain surgery, drugs, etc.\markdownRendererUlItemEnd 
\markdownRendererUlItem Your behavior is a result of neurotransmission, brain structures, etc.\markdownRendererUlItemEnd 
\markdownRendererUlItem When drugs came out, they were supposed to get people out of poor mental states.\markdownRendererUlItemEnd 
\markdownRendererUlEndTight \markdownRendererInterblockSeparator
{}\markdownRendererHeadingTwo{Cognitive}\markdownRendererInterblockSeparator
{}\markdownRendererUlBeginTight
\markdownRendererUlItem Anything about thoughts lmao\markdownRendererUlItemEnd 
\markdownRendererUlItem How do you interpret what happens.\markdownRendererUlItemEnd 
\markdownRendererUlItem This interpretation then, is going to affect your behavior. If you are getting therapy, you are probably getting cognitive behavioral therapy.\markdownRendererUlItemEnd 
\markdownRendererUlItem Ellis and Beck primarily.\markdownRendererUlItemEnd 
\markdownRendererUlEndTight \markdownRendererInterblockSeparator
{}\markdownRendererHeadingTwo{Structuralists vs Functionalists}\markdownRendererInterblockSeparator
{}\markdownRendererUlBeginTight
\markdownRendererUlItem The first two theories were structuralism and functionalism.\markdownRendererUlItemEnd 
\markdownRendererUlItem Structuralism is known to be a part of experimental psychology.\markdownRendererInterblockSeparator
{}\markdownRendererUlBeginTight
\markdownRendererUlItem Focuses on different brain elements and their capacities.\markdownRendererUlItemEnd 
\markdownRendererUlItem Engaging in self-reflective introspection.\markdownRendererUlItemEnd 
\markdownRendererUlItem Attempted to create a "periodic table" for psychology.\markdownRendererUlItemEnd 
\markdownRendererUlEndTight \markdownRendererUlItemEnd 
\markdownRendererUlItem Functionalism was introduced as a counter argument to structuralism.\markdownRendererInterblockSeparator
{}\markdownRendererUlBeginTight
\markdownRendererUlItem Focuses on the adaptations of human mind to different environments.\markdownRendererUlItemEnd 
\markdownRendererUlEndTight \markdownRendererUlItemEnd 
\markdownRendererUlEndTight \markdownRendererInterblockSeparator
{}\markdownRendererHeadingTwo{Types of Research}\markdownRendererInterblockSeparator
{}\markdownRendererUlBeginTight
\markdownRendererUlItem Case Studies\markdownRendererInterblockSeparator
{}\markdownRendererUlBeginTight
\markdownRendererUlItem Study of an individual, or some group. Depends, but it should be very small.\markdownRendererUlItemEnd 
\markdownRendererUlItem Look at why the person became the way that they did.\markdownRendererUlItemEnd 
\markdownRendererUlItem Weak in the regard that the sample size is very small.\markdownRendererUlItemEnd 
\markdownRendererUlItem A bias in the case study is that it can be heavily biased by who is writing it.\markdownRendererUlItemEnd 
\markdownRendererUlEndTight \markdownRendererUlItemEnd 
\markdownRendererUlItem Naturalistic Observation\markdownRendererInterblockSeparator
{}\markdownRendererUlBeginTight
\markdownRendererUlItem Watching humans/animals in the natural environment without interfering.\markdownRendererUlItemEnd 
\markdownRendererUlItem You can't really ask why, otherwise risk screwing up the whole principle of "observation."\markdownRendererUlItemEnd 
\markdownRendererUlEndTight \markdownRendererUlItemEnd 
\markdownRendererUlItem Survey\markdownRendererInterblockSeparator
{}\markdownRendererUlBeginTight
\markdownRendererUlItem The strength of a survey is that you get a lot of information cheaply. Although, framing is a big problem; who you ask; people don't like long surveys, etc.\markdownRendererUlItemEnd 
\markdownRendererUlEndTight \markdownRendererUlItemEnd 
\markdownRendererUlItem Experiment\markdownRendererInterblockSeparator
{}\markdownRendererUlBeginTight
\markdownRendererUlItem The only one that shows cause and effect\markdownRendererUlItemEnd 
\markdownRendererUlItem Contrived behavior, labratory like.\markdownRendererUlItemEnd 
\markdownRendererUlItem When dealing with humans, they may show bias just because it's in a laboratory.\markdownRendererUlItemEnd 
\markdownRendererUlItem Already biased, since most of the research is done on unsuspecting undergraduates.\markdownRendererUlItemEnd 
\markdownRendererUlItem There are ethical things with the experiments.\markdownRendererUlItemEnd 
\markdownRendererUlEndTight \markdownRendererUlItemEnd 
\markdownRendererUlItem Correlational\markdownRendererInterblockSeparator
{}\markdownRendererUlBeginTight
\markdownRendererUlItem Correlation does not imply causation.\markdownRendererUlItemEnd 
\markdownRendererUlItem Be careful for confounding variables.\markdownRendererUlItemEnd 
\markdownRendererUlEndTight \markdownRendererUlItemEnd 
\markdownRendererUlEndTight \markdownRendererInterblockSeparator
{}\markdownRendererHeadingTwo{Sampling}\markdownRendererInterblockSeparator
{}This is literally just AP statistics so you don't have to review this. However, I'm going to take some notes anyways.\markdownRendererInterblockSeparator
{}\markdownRendererUlBeginTight
\markdownRendererUlItem Stratified sample $\rightarrow$ divide into several groups and then take random samples from each.\markdownRendererUlItemEnd 
\markdownRendererUlEndTight \markdownRendererInterblockSeparator
{}\markdownRendererHeadingTwo{Ethics}\markdownRendererInterblockSeparator
{}\markdownRendererUlBegin
\markdownRendererUlItem Confidentiality\markdownRendererUlItemEnd 
\markdownRendererUlItem Informed consent\markdownRendererUlItemEnd 
\markdownRendererUlItem Voluntary Participation\markdownRendererUlItemEnd 
\markdownRendererUlItem Deception (is allowed)\markdownRendererUlItemEnd 
\markdownRendererUlItem Withdrawal Rights\markdownRendererUlItemEnd 
\markdownRendererUlItem Debriefing\markdownRendererInterblockSeparator
{}\markdownRendererUlBeginTight
\markdownRendererUlItem Before they leave, they need to know what they have done.\markdownRendererUlItemEnd 
\markdownRendererUlEndTight \markdownRendererUlItemEnd 
\markdownRendererUlItem Statistically significant, the likelihood that two or more variables is caused by something other than random chance.\markdownRendererUlItemEnd 
\markdownRendererUlEnd \markdownRendererInterblockSeparator
{}\markdownRendererHeadingThree{Normal Distribution}\markdownRendererInterblockSeparator
{}Approximately 65 - 95 - 99.7\%\markdownRendererInterblockSeparator
{}\markdownRendererHeadingOne{Biological Bases of behavior}\markdownRendererInterblockSeparator
{}\markdownRendererUlBeginTight
\markdownRendererUlItem Over 95\% of the neurons we have are interneurons.\markdownRendererUlItemEnd 
\markdownRendererUlItem Afferent take from the brain to the senses.\markdownRendererUlItemEnd 
\markdownRendererUlItem Efferent take from the senses to the brain.\markdownRendererUlItemEnd 
\markdownRendererUlItem Dendrites receive information\markdownRendererUlItemEnd 
\markdownRendererUlItem Soma is the cell body\markdownRendererUlItemEnd 
\markdownRendererUlItem Myelin sheath covers the axon of some neurons and helps speed neural impulses.\markdownRendererUlItemEnd 
\markdownRendererUlItem Dendrites receive messages from other cells.\markdownRendererUlItemEnd 
\markdownRendererUlItem The neurotransmission has inhibitors or exciitors, which help inhibit and excite respectively.\markdownRendererUlItemEnd 
\markdownRendererUlItem Action potential\markdownRendererInterblockSeparator
{}\markdownRendererUlBeginTight
\markdownRendererUlItem A brief electrical charge that travels down the axon of the neuron will fire.\markdownRendererUlItemEnd 
\markdownRendererUlItem Once it hits a negative threshold, then the action potential takes place.\markdownRendererUlItemEnd 
\markdownRendererUlItem Action potential is all or nothing.\markdownRendererUlItemEnd 
\markdownRendererUlEndTight \markdownRendererUlItemEnd 
\markdownRendererUlEndTight \markdownRendererInterblockSeparator
{}\markdownRendererHeadingTwo{Nervous Systems}\markdownRendererInterblockSeparator
{}\markdownRendererUlBegin
\markdownRendererUlItem Sympathetic vs Parasympathetic $\rightarrow$ the para in Parasympathetic stands for parachute which is relaxing. While the sympathetic does quite the opposite, with the fight or flight response.\markdownRendererUlItemEnd 
\markdownRendererUlItem Common neurotransmitters\markdownRendererInterblockSeparator
{}\markdownRendererUlBeginTight
\markdownRendererUlItem Acetylcholine - Attention, arousal, muscle movement, memory; alzheimers.\markdownRendererUlItemEnd 
\markdownRendererUlItem Dopamine - Mood/emotion, arousal, learning; parkinson's, schizophrenia, cocaine & amphetamines. Too much is schizophrenia, too little is parkinson's.\markdownRendererUlItemEnd 
\markdownRendererUlItem Norepinephrine - Mood, arousal, learning; depression. Acronym: Epipen.\markdownRendererUlItemEnd 
\markdownRendererUlItem Serotonin - Sleep hunger, agression, arousal; depression, anxiety, inhibit dreaming.\markdownRendererUlItemEnd 
\markdownRendererUlItem Gaba - slows everything - anxiety, huntington's epilepsey.\markdownRendererUlItemEnd 
\markdownRendererUlEndTight \markdownRendererUlItemEnd 
\markdownRendererUlItem Agonists mimic a drugs.\markdownRendererUlItemEnd 
\markdownRendererUlItem Antagonists blocks a drugs.\markdownRendererUlItemEnd 
\markdownRendererUlEnd \markdownRendererInterblockSeparator
{}\markdownRendererHeadingTwo{Parts of the Brain}\markdownRendererInterblockSeparator
{}\markdownRendererUlBeginTight
\markdownRendererUlItem Medulla does the dull, automatic things; breathing, heart breathing.\markdownRendererUlItemEnd 
\markdownRendererUlItem Pons - restful sleeping, like a pond. Sleep and wakefulness.\markdownRendererUlItemEnd 
\markdownRendererUlItem Cerebellum - Muscle movement and coordination; think of Sarah ringing a bell jumping around the room like a ballerina telling us about how coordinated she is.\markdownRendererUlItemEnd 
\markdownRendererUlItem Reticular activating system - arousal\markdownRendererUlItemEnd 
\markdownRendererUlItem Thalamus - looks like an egg in the middle of the brain - relays things into the upper part of the brain.\markdownRendererUlItemEnd 
\markdownRendererUlItem Hypothalamus - gets you ready for fight or flight, feeding.\markdownRendererUlItemEnd 
\markdownRendererUlItem Amygdala - emotional amy - emotions, particularly fear.\markdownRendererUlItemEnd 
\markdownRendererUlItem Hippocampus - associated with memory - hippo on campus about memroy.\markdownRendererUlItemEnd 
\markdownRendererUlItem Pituitary gland - in limbic system.\markdownRendererUlItemEnd 
\markdownRendererUlItem Cerebrum, Cortex, etc. About 70\% of the human brain.\markdownRendererUlItemEnd 
\markdownRendererUlItem The corpus collosum connects the two hemispheres. In epilepsey, this is cut.\markdownRendererUlItemEnd 
\markdownRendererUlItem Endocrine is practically controlled by the pituitary gland. Mainly glands.\markdownRendererUlItemEnd 
\markdownRendererUlEndTight \markdownRendererInterblockSeparator
{}\markdownRendererHeadingTwo{Lobes of the Brain}\markdownRendererInterblockSeparator
{}\markdownRendererUlBegin
\markdownRendererUlItem Occipital - vision\markdownRendererUlItemEnd 
\markdownRendererUlItem Temporal - Memory, understanding, time, language.\markdownRendererUlItemEnd 
\markdownRendererUlItem Parietal Lobe - Senses.\markdownRendererUlItemEnd 
\markdownRendererUlItem Motor cortex - Movement.\markdownRendererUlItemEnd 
\markdownRendererUlItem Frontal lobe - executive functions, thinking, planning, organizing, problem sovling emotions and behavioroual control - personality.\markdownRendererUlItemEnd 
\markdownRendererUlItem Broca's Area - produces speech\markdownRendererUlItemEnd 
\markdownRendererUlItem Wernicke's area - understands speech.\markdownRendererUlItemEnd 
\markdownRendererUlItem Plasticity is saying that other parts of the brain will take over if other parts are damaged.\markdownRendererUlItemEnd 
\markdownRendererUlItem Neurogenesis is that you can get new nerve cells.\markdownRendererUlItemEnd 
\markdownRendererUlItem Absolute threshold - The weakest amount of a stimulus that a person can detect 50\% of the time.\markdownRendererUlItemEnd 
\markdownRendererUlItem Weber's Law: difference threshold is proportional.\markdownRendererUlItemEnd 
\markdownRendererUlItem Cocktail Party Effect: Do you understand that you can pull out one stimulus even though there are a bunch going around you, or there's something someone says your name. It's the ability to focus on one voice, or listening to something and it grabs your attention.\markdownRendererUlItemEnd 
\markdownRendererUlItem Signal detection theory - Absolute threshold is fluid. Hit, false alarm, miss, correct rejection.\markdownRendererUlItemEnd 
\markdownRendererUlEnd \markdownRendererInterblockSeparator
{}\markdownRendererHeadingTwo{Sensation and Perception}\markdownRendererInterblockSeparator
{}\markdownRendererUlBeginTight
\markdownRendererUlItem Cornea, it's the front part of the eye.\markdownRendererUlItemEnd 
\markdownRendererUlItem Iris, is the colored part, has muscle that controls the pupil which allows light in.\markdownRendererUlItemEnd 
\markdownRendererUlItem Lens accommodates for that light and flips the image over.\markdownRendererUlItemEnd 
\markdownRendererUlItem Back part of the eye: retina, has rods and cones.\markdownRendererUlItemEnd 
\markdownRendererUlItem Bipolar cell.\markdownRendererUlItemEnd 
\markdownRendererUlItem All of these leave the eye through the optic nerve. The eye has a blind spot.\markdownRendererUlItemEnd 
\markdownRendererUlItem Monocoular cues.\markdownRendererInterblockSeparator
{}\markdownRendererUlBeginTight
\markdownRendererUlItem Perspective is the effect of creating depth.\markdownRendererUlItemEnd 
\markdownRendererUlEndTight \markdownRendererUlItemEnd 
\markdownRendererUlItem Gestalt Principles\markdownRendererInterblockSeparator
{}\markdownRendererUlBeginTight
\markdownRendererUlItem Ways we organize the world.\markdownRendererUlItemEnd 
\markdownRendererUlItem Proximity, similiarity, enclosure, symmetry, closure, contunuity, connection, figure & ground.\markdownRendererUlItemEnd 
\markdownRendererUlEndTight \markdownRendererUlItemEnd 
\markdownRendererUlItem Tastes\markdownRendererInterblockSeparator
{}\markdownRendererUlBeginTight
\markdownRendererUlItem Sour, Bitter, Salty, Sweet, Umami\markdownRendererUlItemEnd 
\markdownRendererUlEndTight \markdownRendererUlItemEnd 
\markdownRendererUlItem Top-down processing is going from the big picture to looking at the smaller parts.\markdownRendererUlItemEnd 
\markdownRendererUlItem Bottom-up processing going from the smaller pieces into the bigger ones.\markdownRendererUlItemEnd 
\markdownRendererUlEndTight \markdownRendererInterblockSeparator
{}\markdownRendererHeadingTwo{Theories of Color}\markdownRendererInterblockSeparator
{}\markdownRendererUlBeginTight
\markdownRendererUlItem Trichromatic theory, there is a red green blue signals (all separate).\markdownRendererUlItemEnd 
\markdownRendererUlItem Opponent process theory, red signal and green signal go to brain, blue and yellow go to brain.\markdownRendererUlItemEnd 
\markdownRendererUlItem Combined theory: half and half of the above two.\markdownRendererUlItemEnd 
\markdownRendererUlEndTight \markdownRendererInterblockSeparator
{}\markdownRendererHeadingTwo{Stages of Consciousness}\markdownRendererInterblockSeparator
{}\markdownRendererUlBeginTight
\markdownRendererUlItem Sleep waves (BATSD)\markdownRendererInterblockSeparator
{}\markdownRendererUlBeginTight
\markdownRendererUlItem Beta - awake\markdownRendererUlItemEnd 
\markdownRendererUlItem Alpha - relaxed\markdownRendererUlItemEnd 
\markdownRendererUlItem Theta - Stage one sleep\markdownRendererUlItemEnd 
\markdownRendererUlItem Spindles, k-complexes - Stage two sleep\markdownRendererUlItemEnd 
\markdownRendererUlItem Delta waves - Stage three sleep.\markdownRendererUlItemEnd 
\markdownRendererUlEndTight \markdownRendererUlItemEnd 
\markdownRendererUlItem Rem sleep gets longer through the night.\markdownRendererUlItemEnd 
\markdownRendererUlItem Rem deprivation where you wake them up right before REM, you are more likely to go to REM sleep quicker.\markdownRendererUlItemEnd 
\markdownRendererUlEndTight \markdownRendererInterblockSeparator
{}\markdownRendererHeadingTwo{Drugs}\markdownRendererInterblockSeparator
{}\markdownRendererUlBegin
\markdownRendererUlItem Stimulants: excite the nervous system.\markdownRendererInterblockSeparator
{}\markdownRendererUlBeginTight
\markdownRendererUlItem Cocaine\markdownRendererUlItemEnd 
\markdownRendererUlItem Nicotine\markdownRendererUlItemEnd 
\markdownRendererUlItem Amphetamine\markdownRendererUlItemEnd 
\markdownRendererUlEndTight \markdownRendererUlItemEnd 
\markdownRendererUlItem Depressants: slow the nervous system.\markdownRendererInterblockSeparator
{}\markdownRendererUlBeginTight
\markdownRendererUlItem Xanax\markdownRendererUlItemEnd 
\markdownRendererUlItem Alcohol\markdownRendererUlItemEnd 
\markdownRendererUlItem Opioids\markdownRendererUlItemEnd 
\markdownRendererUlItem Oxycotin\markdownRendererUlItemEnd 
\markdownRendererUlItem Heroin\markdownRendererUlItemEnd 
\markdownRendererUlItem LSD\markdownRendererUlItemEnd 
\markdownRendererUlEndTight \markdownRendererUlItemEnd 
\markdownRendererUlItem Dream theories\markdownRendererInterblockSeparator
{}\markdownRendererUlBeginTight
\markdownRendererUlItem Freud's Wish-fulfillment\markdownRendererUlItemEnd 
\markdownRendererUlItem Information-processing - dreams help us sort out the day's events.\markdownRendererUlItemEnd 
\markdownRendererUlItem Physiological function - Helps preserve neural path ways.\markdownRendererUlItemEnd 
\markdownRendererUlItem Activation-synthesis - your mind is active, and your dreams help synthesize your information.\markdownRendererUlItemEnd 
\markdownRendererUlEndTight \markdownRendererUlItemEnd 
\markdownRendererUlEnd \markdownRendererInterblockSeparator
{}\markdownRendererHeadingTwo{Learning}\markdownRendererInterblockSeparator
{}\markdownRendererUlBeginTight
\markdownRendererUlItem Classical conditioning\markdownRendererUlItemEnd 
\markdownRendererUlItem Operant conditioning\markdownRendererUlItemEnd 
\markdownRendererUlItem Pavlov's experiment\markdownRendererInterblockSeparator
{}\markdownRendererUlBeginTight
\markdownRendererUlItem Used a sound which was a neutral stimulus. If you do the tuning fork just before you present the bell, the dog salivates.\markdownRendererUlItemEnd 
\markdownRendererUlEndTight \markdownRendererUlItemEnd 
\markdownRendererUlItem Reinforcers\markdownRendererInterblockSeparator
{}\markdownRendererUlBeginTight
\markdownRendererUlItem Primary - biological needs such as food\markdownRendererUlItemEnd 
\markdownRendererUlItem Secondary - money, etc.\markdownRendererUlItemEnd 
\markdownRendererUlEndTight \markdownRendererUlItemEnd 
\markdownRendererUlItem Negative - taking away\markdownRendererUlItemEnd 
\markdownRendererUlItem Positive - giving something\markdownRendererUlItemEnd 
\markdownRendererUlItem Negative Reinforcement - Parents nag you to take out the trash. Negative thoughts.\markdownRendererUlItemEnd 
\markdownRendererUlItem Positive punishment - someone hits you. Something does something to you. Introduces an unpleasent stimulus\markdownRendererUlItemEnd 
\markdownRendererUlItem Negative punishment - takes away something you want like a cell phone.\markdownRendererUlItemEnd 
\markdownRendererUlItem Schedules of Reinforcement\markdownRendererInterblockSeparator
{}\markdownRendererUlBeginTight
\markdownRendererUlItem Ratio happens after a certain number of responses.\markdownRendererUlItemEnd 
\markdownRendererUlItem Interval is about waiting.\markdownRendererUlItemEnd 
\markdownRendererUlEndTight \markdownRendererUlItemEnd 
\markdownRendererUlItem Escape learning - leave to avoid a stimulus.\markdownRendererUlItemEnd 
\markdownRendererUlItem Avoidance learning - leave to avoid a stimulus.\markdownRendererUlItemEnd 
\markdownRendererUlItem Learned helplessness - perceived lack of control which forces a generalized helpless behavior.\markdownRendererUlItemEnd 
\markdownRendererUlItem Overjustifcation effect - give too much award and lead somethign to become an extrinsic reward.\markdownRendererUlItemEnd 
\markdownRendererUlEndTight \markdownRendererInterblockSeparator
{}\markdownRendererHeadingTwo{Memory}\markdownRendererInterblockSeparator
{}\markdownRendererUlBeginTight
\markdownRendererUlItem Explicit memory - also known as declarative memory, with conscious recall.\markdownRendererUlItemEnd 
\markdownRendererUlItem Implicit memory - without conscious recall.\markdownRendererUlItemEnd 
\markdownRendererUlItem Priming\markdownRendererUlItemEnd 
\markdownRendererUlItem Procedural memory - motor skill and congitive skils.\markdownRendererUlItemEnd 
\markdownRendererUlItem Iconic memory - visual information that lasts around 0.5 seconds.\markdownRendererUlItemEnd 
\markdownRendererUlItem Echoic Memory - Auditory information that lasts several seconds 3-4.\markdownRendererUlItemEnd 
\markdownRendererUlItem Flashbulb Memory - Strong, seemingly very accurate memory made during a shocking event.\markdownRendererUlItemEnd 
\markdownRendererUlItem Prospective memory: remembering to do future tasks.\markdownRendererUlItemEnd 
\markdownRendererUlEndTight \markdownRendererInterblockSeparator
{}\markdownRendererHeadingOne{Personality}\markdownRendererInterblockSeparator
{}\markdownRendererHeadingTwo{Freud}\markdownRendererInterblockSeparator
{}\markdownRendererUlBeginTight
\markdownRendererUlItem Ego -\markdownRendererUlItemEnd 
\markdownRendererUlEndTight \relax