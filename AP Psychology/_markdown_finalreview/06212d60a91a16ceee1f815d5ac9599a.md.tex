\markdownRendererHeadingOne{Research and History}\markdownRendererInterblockSeparator
{}\markdownRendererHeadingTwo{Psychoanalytic Approach}\markdownRendererInterblockSeparator
{}\markdownRendererUlBeginTight
\markdownRendererUlItem Very influenced by Sigmeund Freud.\markdownRendererUlItemEnd 
\markdownRendererUlItem In the late 1800s, he came up with a psychoanalytic approach which was about the unconscious drives.\markdownRendererUlItemEnd 
\markdownRendererUlItem For the most part, psychological approach were about drawing apart these unconscious desires.\markdownRendererUlItemEnd 
\markdownRendererUlEndTight \markdownRendererInterblockSeparator
{}\markdownRendererHeadingTwo{Behvaioral}\markdownRendererInterblockSeparator
{}\markdownRendererUlBeginTight
\markdownRendererUlItem Later, Ivan Pavlov, Skinner, Watson, theorized that you were all influenced about the ways that you were reinforced/punished.\markdownRendererUlItemEnd 
\markdownRendererUlEndTight \markdownRendererInterblockSeparator
{}\markdownRendererHeadingTwo{Humanist}\markdownRendererInterblockSeparator
{}\markdownRendererUlBeginTight
\markdownRendererUlItem Stemmed from hippies.\markdownRendererUlItemEnd 
\markdownRendererUlItem Humanist is the drive for free will.\markdownRendererUlItemEnd 
\markdownRendererUlItem You are trying to drive to reach your greatest free potential.\markdownRendererUlItemEnd 
\markdownRendererUlEndTight \markdownRendererInterblockSeparator
{}\markdownRendererBlockQuoteBegin
Mr. Rogers, isn't it a special day today
\markdownRendererBlockQuoteEnd \markdownRendererInterblockSeparator
{}\markdownRendererUlBeginTight
\markdownRendererUlItem Things like uncoditional positive regard, you are the best person to solve your own problem.\markdownRendererUlItemEnd 
\markdownRendererUlEndTight \markdownRendererInterblockSeparator
{}\markdownRendererHeadingTwo{Biological}\markdownRendererInterblockSeparator
{}\markdownRendererUlBeginTight
\markdownRendererUlItem Brain surgery, drugs, etc.\markdownRendererUlItemEnd 
\markdownRendererUlItem Your behavior is a result of neurotransmission, brain structures, etc.\markdownRendererUlItemEnd 
\markdownRendererUlItem When drugs came out, they were supposed to get people out of poor mental states.\markdownRendererUlItemEnd 
\markdownRendererUlEndTight \markdownRendererInterblockSeparator
{}\markdownRendererHeadingTwo{Cognitive}\markdownRendererInterblockSeparator
{}\markdownRendererUlBeginTight
\markdownRendererUlItem Anything about thoughts lmao\markdownRendererUlItemEnd 
\markdownRendererUlItem How do you interpret what happens.\markdownRendererUlItemEnd 
\markdownRendererUlItem This interpretation then, is going to affect your behavior. If you are getting therapy, you are probably getting cognitive behavioral therapy.\markdownRendererUlItemEnd 
\markdownRendererUlItem Ellis and Beck primarily.\markdownRendererUlItemEnd 
\markdownRendererUlEndTight \markdownRendererInterblockSeparator
{}\markdownRendererHeadingTwo{Structuralists vs Functionalists}\markdownRendererInterblockSeparator
{}\markdownRendererUlBeginTight
\markdownRendererUlItem The first two theories were structuralism and functionalism.\markdownRendererUlItemEnd 
\markdownRendererUlItem Structuralism is known to be a part of experimental psychology.\markdownRendererInterblockSeparator
{}\markdownRendererUlBeginTight
\markdownRendererUlItem Focuses on different brain elements and their capacities.\markdownRendererUlItemEnd 
\markdownRendererUlEndTight \markdownRendererUlItemEnd 
\markdownRendererUlItem Functionalism was introduced as a counter argument to structuralism.\markdownRendererInterblockSeparator
{}\markdownRendererUlBeginTight
\markdownRendererUlItem Focuses on the adaptations of human mind to different environments.\markdownRendererUlItemEnd 
\markdownRendererUlEndTight \markdownRendererUlItemEnd 
\markdownRendererUlEndTight \markdownRendererInterblockSeparator
{}\markdownRendererHeadingTwo{Types of Research}\markdownRendererInterblockSeparator
{}\markdownRendererUlBeginTight
\markdownRendererUlItem Case Studies\markdownRendererInterblockSeparator
{}\markdownRendererUlBeginTight
\markdownRendererUlItem Study of an individual, or some group. Depends, but it should be very small.\markdownRendererUlItemEnd 
\markdownRendererUlItem Look at why the person became the way that they did.\markdownRendererUlItemEnd 
\markdownRendererUlItem Weak in the regard that the sample size is very small.\markdownRendererUlItemEnd 
\markdownRendererUlItem A bias in the case study is that it can be heavily biased by who is writing it.\markdownRendererUlItemEnd 
\markdownRendererUlEndTight \markdownRendererUlItemEnd 
\markdownRendererUlItem Naturalistic Observation\markdownRendererInterblockSeparator
{}\markdownRendererUlBeginTight
\markdownRendererUlItem Watching humans/animals in the natural environment without interfering.\markdownRendererUlItemEnd 
\markdownRendererUlItem You can't really ask why, otherwise risk screwing up the whole principle of "observation."\markdownRendererUlItemEnd 
\markdownRendererUlEndTight \markdownRendererUlItemEnd 
\markdownRendererUlItem Survey\markdownRendererInterblockSeparator
{}\markdownRendererUlBeginTight
\markdownRendererUlItem The strength of a survey is that you get a lot of information cheaply. Although, framing is a big problem; who you ask; people don't like long surveys, etc.\markdownRendererUlItemEnd 
\markdownRendererUlEndTight \markdownRendererUlItemEnd 
\markdownRendererUlItem Experiment\markdownRendererInterblockSeparator
{}\markdownRendererUlBeginTight
\markdownRendererUlItem The only one that shows cause and effect\markdownRendererUlItemEnd 
\markdownRendererUlItem Contrived behavior, labratory like.\markdownRendererUlItemEnd 
\markdownRendererUlItem When dealing with humans, they may show bias just because it's in a laboratory.\markdownRendererUlItemEnd 
\markdownRendererUlItem Already biased, since most of the research is done on unsuspecting undergraduates.\markdownRendererUlItemEnd 
\markdownRendererUlItem There are ethical things with the experiments.\markdownRendererUlItemEnd 
\markdownRendererUlEndTight \markdownRendererUlItemEnd 
\markdownRendererUlItem Correlational\markdownRendererInterblockSeparator
{}\markdownRendererUlBeginTight
\markdownRendererUlItem Correlation does not imply causation.\markdownRendererUlItemEnd 
\markdownRendererUlItem Be careful for confounding variables.\markdownRendererUlItemEnd 
\markdownRendererUlEndTight \markdownRendererUlItemEnd 
\markdownRendererUlEndTight \relax