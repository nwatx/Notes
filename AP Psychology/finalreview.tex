\documentclass{article}
\usepackage{amsmath}
\usepackage[smartEllipses,hybrid]{markdown}
\usepackage[margin=0.5in]{geometry}
\usepackage{enumitem}
\usepackage{hyperref}
\usepackage{parskip}
% \setlist[itemize]{align=parleft,left=0pt..1em}
% \setlist[enumerate]{align=parleft,left=0pt..1em}
\renewcommand{\labelitemii}{$\circ$}
\renewcommand{\labelitemiii}{$\circ$}

\title{AP Psychology Final Review}
\author{Neo Wang}
\date{\today}

\begin{document}

\maketitle

Before we begin, this should probably serve as a heads up warning on why you shouldn't start studying for the AP exams three days before they happen.

\tableofcontents

\begin{markdown}

# Research and History

## Psychoanalytic Approach

* Very influenced by Sigmeund Freud.
* In the late 1800s, he came up with a psychoanalytic approach which was about the unconscious drives.
* For the most part, psychological approach were about drawing apart these unconscious desires.

## Behvaioral
* Later, Ivan Pavlov, Skinner, Watson, theorized that you were all influenced about the ways that you were reinforced/punished.

## Humanist

* Stemmed from hippies.
* Humanist is the drive for free will.
* You are trying to drive to reach your greatest free potential.

> Mr. Rogers, isn't it a special day today

* Things like uncoditional positive regard, you are the best person to solve your own problem.

## Biological

* Brain surgery, drugs, etc.
* Your behavior is a result of neurotransmission, brain structures, etc.
* When drugs came out, they were supposed to get people out of poor mental states.

## Cognitive

* Anything about thoughts lmao
* How do you interpret what happens.
* This interpretation then, is going to affect your behavior. If you are getting therapy, you are probably getting cognitive behavioral therapy.
* Ellis and Beck primarily.

## Structuralists vs Functionalists

* The first two theories were structuralism and functionalism.
* Structuralism is known to be a part of experimental psychology.
	* Focuses on different brain elements and their capacities.
* Functionalism was introduced as a counter argument to structuralism.
	* Focuses on the adaptations of human mind to different environments.

## Types of Research

* Case Studies
	* Study of an individual, or some group. Depends, but it should be very small.
	* Look at why the person became the way that they did.
	* Weak in the regard that the sample size is very small.
	* A bias in the case study is that it can be heavily biased by who is writing it.
* Naturalistic Observation
	* Watching humans/animals in the natural environment without interfering.
	* You can't really ask why, otherwise risk screwing up the whole principle of "observation."
* Survey
	* The strength of a survey is that you get a lot of information cheaply. Although, framing is a big problem; who you ask; people don't like long surveys, etc.
* Experiment
	* The only one that shows cause and effect
	* Contrived behavior, labratory like.
	* When dealing with humans, they may show bias just because it's in a laboratory.
	* Already biased, since most of the research is done on unsuspecting undergraduates.
	* There are ethical things with the experiments.
* Correlational
	* Correlation does not imply causation.
	* Be careful for confounding variables.

## Sampling

This is literally just AP statistics so you don't have to review this. However, I'm going to take some notes anyways.

* Stratified sample $\rightarrow$ divide into several groups and then take random samples from each.

## Ethics

* Confidentiality
* Informed consent
* Voluntary Participation
* Deception (is allowed)
* Withdrawal Rights
* Debriefing
	* Before they leave, they need to know what they have done.

* Statistically significant, the likelihood that two or more variables is caused by something other than random chance.

### Normal Distribution

Approximately 65 - 95 - 99.7\%

# Biological Bases of behavior

* Over 95\% of the neurons we have are interneurons.
* Afferent take from the brain to the senses.
* Efferent take from the senses to the brain.
* Dendrites receive information
* Soma is the cell body
* Myelin sheath covers the axon of some neurons and helps speed neural impulses.
* Dendrites receive messages from other cells.
* The neurotransmission has inhibitors or exciitors, which help inhibit and excite respectively.
* Action potential
	* A brief electrical charge that travels down the axon of the neuron will fire.
	* Once it hits a negative threshold, then the action potential takes place.
	* Action potential is all or nothing.

## Nervous Systems

* Sympathetic vs Parasympathetic $\rightarrow$ the para in Parasympathetic stands for parachute which is relaxing. While the sympathetic does quite the opposite, with the fight or flight response.

* Common neurotransmitters
	* Acetylcholine - Attention, arousal, muscle movement, memory; alzheimers.
	* Dopamine - Mood/emotion, arousal, learning; parkinson's, schizophrenia, cocaine & amphetamines. Too much is schizophrenia, too little is parkinson's.
	* Norepinephrine - Mood, arousal, learning; depression. Acronym: Epipen.
	* Serotonin - Sleep hunger, agression, arousal; depression, anxiety, inhibit dreaming.
	* Gaba - slows everything - anxiety, huntington's epilepsey.

* Agonists mimic a drugs.
* Antagonists blocks a drugs.

## Parts of the Brain

* Medulla does the dull, automatic things; breathing, heart breathing.
* Pons - restful sleeping, like a pond. Sleep and wakefulness.
* Cerebellum - Muscle movement and coordination; think of Sarah ringing a bell jumping around the room like a ballerina telling us about how coordinated she is.
* Reticular activating system - arousal
* Thalamus - looks like an egg in the middle of the brain - relays things into the upper part of the brain.
* Hypothalamus - gets you ready for fight or flight, feeding.
* Amygdala - emotional amy - emotions, particularly fear.
* Hippocampus - associated with memory - hippo on campus about memroy.
* Pituitary gland - in limbic system.
* Cerebrum, Cortex, etc. About 70\% of the human brain.
* The corpus collosum connects the two hemispheres. In epilepsey, this is cut.
* Endocrine is practically controlled by the pituitary gland. Mainly glands.

## Lobes of the Brain

* Occipital - vision
* Temporal - Memory, understanding, time, language.
* Parietal Lobe - Senses.
* Motor cortex - Movement.
* Frontal lobe - executive functions, thinking, planning, organizing, problem sovling emotions and behavioroual control - personality.
* Broca's Area - produces speech
* Wernicke's area - understands speech.

* Plasticity is saying that other parts of the brain will take over if other parts are damaged.
* Neurogenesis is that you can get new nerve cells.

* Absolute threshold - The weakest amount of a stimulus that a person can detect 50\% of the time.
* Weber's Law: difference threshold is proportional.

\end{markdown}

\end{document}